\documentclass[12pt]{article}

\title{WE2: Elektrische motoren (groep 2\_5)}
\author{\textbf{Dries Van den Brande} \and Wassim Amghizar 
\and Andreas Declerck \and Joren Vermeersch}


\begin{document}

\maketitle

\section{Doel}
Het doel van dit practica is experimenteel bepalen van het rendement
van een gelijkstroommotor en wisselstroomgenerator

\section{Materiaal}
\begin{itemize}
    \item Gelijkstroommotor
    \item Generator
    \item RPM-meter
    \item Elektrodynamische vermogenmeter
    \item Stroomtransformator
\end{itemize}

\section{Formules}
De opstelling was uitgerust met een gelijkstroommotor. Hierbij gebeurt de bepaling van $P_{el1}$
aan de hand van de meting van de gelijkspanning waarmee de motor wordt gevoed $U_1$ en de stroom 
die hij opslorpt $I_1$. Het vermogen is gelijk aan het product van de stroom en de spanning:
\begin{equation}
    P_{el1} = U_1 \cdot I_1
\end{equation}
Hiernaast was de proefopstelling uitgerust met alternatoren die 
elektrische energie opwekken onder de vorm van wisselspanning. De meting van $P_{el2}$
gebeurt met een wisselstroom wattmeter. De assistent liet ons weten dat we bij de vermogenmeter
niet de spanning en stroom moeten noteren.\\

\noindent Het mechanisch vermogen waarmee de motor de generator aandrijft is te berekenen
met de formule: 
\begin{equation}
    P_{mech} = M \cdot n \cdot \frac{2\pi}{60}
\end{equation} 
met \\
M: koppel gemeten met de koppelmetern: hoeksnelheid in frac{toeren}{min}

\section{Foutenanalyse}
\subsection{Systematische fouten}

\subsection{Toevallige fouten}
\textbf{Rechtstreekse grootheden}\\
De $U_1,I_1,M,n$ en $P_{el2}$ werden rechtstreeks gemeten. De toevallige fouten
op deze grootheden werden berekend aan de hand van de volgende formule: 
\begin{equation}
    \sigma=\sqrt{\frac{\sum\limits_{i=0}^n(x_i-m)^2}{n-1}}  
\end{equation}

\noindent \textbf{Onrechtstreekse grootheden}\\
De onrechtstreekse grootheden in dit practica zijn $P_{el1},P_{mech},\eta_{1}$ en
$\eta_2$.\\
Volgende formules werden gebruikt om de toevallige fouten te bepalen:\\
\begin{equation}
    \sigma^2\{R\}=\sum\limits_{i=0}^n\left(\frac{\partial \Psi}{\partial X_i}\right)_0^2 \sigma ^2 \{X_i\}
\end{equation}

\end{document}