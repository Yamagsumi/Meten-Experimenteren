\section{Foutenanalyse}
\subsection{Systematische fouten}

\subsection{Toevallige fouten}
\textbf{Rechtstreekse grootheden}\\
De $U_1,I_1,M,n$ en $P_{el2}$ werden rechtstreeks gemeten. De toevallige fouten
op deze grootheden werden berekend aan de hand van de volgende formule: 
\begin{equation}
    \sigma=\sqrt{\frac{\sum\limits_{i=0}^n(x_i-m)^2}{n-1}}  
\end{equation}

\noindent \textbf{Onrechtstreekse grootheden}\\
De onrechtstreekse grootheden in dit practica zijn $P_{el1},P_{mech},\eta_{1}$ en
$\eta_2$.\\
Volgende formules werden gebruikt om de toevallige fouten te bepalen:\\
\begin{equation}
    \sigma^2\{R\}=\sum\limits_{i=0}^n\left(\frac{\partial \Psi}{\partial X_i}\right)_0^2 \sigma ^2 \{X_i\}
\end{equation}